\documentclass[xcolor=x11names]{beamer}
\usepackage[utf8]{inputenc}
\usepackage[T1]{fontenc}
\usepackage{graphicx}
%\usepackage{verbatim}
\usepackage{tikz}
\usetikzlibrary{decorations.fractals}
\usetikzlibrary{decorations.text}
\usepgflibrary{arrows}
\usetikzlibrary{fadings}
\usetikzlibrary[decorations.pathmorphing]
\tikzfading[name=fade inside, inner color=transparent!70, outer color=transparent!70]
\usetikzlibrary{calc}
\usetikzlibrary{intersections}
\usetikzlibrary{shapes}
\usetikzlibrary{patterns}
\usefonttheme{serif}
\usepackage{amssymb} 			
\usepackage{amsmath}
\usepackage{ifthen}
\usepackage[normalem]{ulem}
\usepackage{mathrsfs}

%%%%%%%%%%%%%%%%%%%%%%%%%%%%%%%%%%%%%%%%%%%%%%%%%%%%%%%%%%%%%%%%%%%%%%%%%%%%%%%%%%%%
%% Beamer Layout %%%%%%%%%%%%%%%%%%%%%%%%%%%%%%%%%%
\useoutertheme[subsection=false,shadow]{miniframes}
\useinnertheme{default}
\usefonttheme{serif}
%\usepackage{txfonts} %Hook for strict implication!
\DeclareSymbolFont{symbolsC}{U}{txsyc}{m}{n}
\DeclareMathSymbol{\strictif}{\mathrel}{symbolsC}{74}
\DeclareMathSymbol{\boxright}{\mathrel}{symbolsC}{128}
\usepackage{palatino}
%\usepackage[uppercase=upright,charter]{mathdesign}

\setbeamerfont{title like}{shape=\scshape}
\setbeamerfont{frametitle}{shape=\scshape}


\setbeamercolor*{lower separation line head}{bg=white!40!DeepSkyBlue3}
\setbeamercolor*{normal text}{fg=black,bg=white}
\setbeamercolor*{alerted text}{fg=red}
\setbeamercolor*{example text}{fg=black}
\setbeamercolor*{structure}{fg=black}

\setbeamercolor*{palette tertiary}{fg=black,bg=white!90!DeepSkyBlue3}
\setbeamercolor*{palette quaternary}{fg=black,bg=black!10}

%\setbeamercolor{block body alerted}{bg=normal text.bg!90!DeepSkyBlue4}
\setbeamercolor{block body}{bg=normal text.bg!95!DeepSkyBlue3}
%\setbeamercolor{block body example}{bg=normal text.bg!90!DeepSkyBlue4}
%\setbeamercolor{block title alerted}{use={normal text,alerted text},fg=alerted text.fg!75!normal text.fg,bg=normal text.bg!90!DeepSkyBlue4}
\setbeamercolor{block title}{bg=normal text.bg!70!DeepSkyBlue3}
%\setbeamercolor{block title example}{use={normal text,example text},fg=example text.fg!75!normal text.fg,bg=normal text.bg!75!DeepSkyBlue4}

\setbeamertemplate{blocks}[rounded][shadow=true]
%\setbeamertemplate{background canvas}[vertical shading][bottom=white,top=structure.fg!25]
%\setbeamertemplate{sidebar canvas left}[horizontal shading][left=white!40!black,right=black]
\setbeamertemplate{itemize items}[circle]
\setbeamercolor*{itemize item}{fg=DeepSkyBlue3}
\setbeamercolor*{itemize subitem}{fg=DeepSkyBlue3}
\setbeamercolor*{itemize subsubitem}{fg=DeepSkyBlue3}
\setbeamertemplate{enumerate items}[circle]
%\setbeamercolor{item projected}{bg=DeepSkyBlue3,fg=black}
\setbeamercolor{item projected}{bg=white,fg=DeepSkyBlue3}
\setbeamercolor*{enumerate item}{fg=DeepSkyBlue3}
\setbeamercolor*{enumerate subitem}{fg=DeepSkyBlue3}
\setbeamercolor*{enumerate subsubitem}{fg=DeepSkyBlue3}

%%%%%%%%%%%%%%%%%%%%%%%%%%%%%%%%%%%%%%%%%%%%%%%%%%


%%%%%%%%%%%%%%%%%%%%%%%%%%%%%%%%%%%%%%%%%%%%%%%%%%%%%%%%%%%%%%%%%%%%%%%%%%%%%%%%%%%%

\newenvironment{defi}[1][]{\begin{block}{\footnotesize \textsc{Definition} \ifthenelse{\equal{#1}{}}{}{\, (#1)}}}{\end{block}}
\newenvironment{prop}[1][]{\begin{block}{\footnotesize \textsc{Proposition} \ifthenelse{\equal{#1}{}}{}{\, (\textsc{#1})}}}{\end{block}}
\newenvironment{lemm}[1][]{\begin{block}{\footnotesize \textsc{Lemma} \ifthenelse{\equal{#1}{}}{}{\, (\textsc{#1})}}}{\end{block}}
\newenvironment{idea}[1][]{\begin{block}{\footnotesize \textsc{Idea} \ifthenelse{\equal{#1}{}}{}{\, (\textsc{#1})}}}{\end{block}}
\newenvironment{rema}[1][]{\begin{block}{\footnotesize \textsc{Remark} \ifthenelse{\equal{#1}{}}{}{\, (\textsc{#1})}}}{\end{block}}
\newenvironment{coro}[1][]{\begin{block}{\footnotesize \textsc{Corollary} \ifthenelse{\equal{#1}{}}{}{\, (\textsc{#1})}}}{\end{block}}
\newenvironment{tete}[1][]{\begin{block}{\footnotesize \textsc{Theorem} \ifthenelse{\equal{#1}{}}{}{\, (\textsc{#1})}}}{\end{block}}
\newenvironment{claim}[1][]{\begin{block}{Claim \ifthenelse{\equal{#1}{}}{}{\, (\textsc{#1})}}}{\end{block}}
%\newenvironment{lemma}[1][]{\begin{block}{Lemma \ifthenelse{\equal{#1}{}}{}{\, (\textsc{#1})}}}{\end{block}}
\newenvironment{question}[1][]{\begin{block}{Question \ifthenelse{\equal{#1}{}}{}{\, (\textsc{#1})}}}{\end{block}}
\newenvironment{rem}[1][]{\begin{block}{Remark \ifthenelse{\equal{#1}{}}{}{\, (\textsc{#1})}}}{\end{block}}
\newenvironment{homework}[1][]{\begin{block}{Homework \ifthenelse{\equal{#1}{}}{}{\, (\textsc{#1})}}}{\end{block}}
\newenvironment{proo}[1][]{\begin{block}{\footnotesize \textsc{Proof} \ifthenelse{\equal{#1}{}}{}{\, (\textsc{#1})}}}{\end{block}}

%%%%%%%%%%%%%%%%%%%%%
%% To evade unnecessary circles, mainly for \cimdia
%%%%%%%%%%%%%%%%%%%%%

\makeatletter
\let\beamer@writeslidentry@miniframeson=\beamer@writeslidentry
\def\beamer@writeslidentry@miniframesoff{%
  \expandafter\beamer@ifempty\expandafter{\beamer@framestartpage}{}% does not happen normally
  {%else
    % removed \addtocontents commands
    \clearpage\beamer@notesactions%
  }
}
\newcommand*{\miniframeson}{\let\beamer@writeslidentry=\beamer@writeslidentry@miniframeson}
\newcommand*{\miniframesoff}{\let\beamer@writeslidentry=\beamer@writeslidentry@miniframesoff}
\makeatother

%%%%%%%%%%%%%%%%%%%%%%%%%%%%%
%%%%%%%%%%%%% END %%%%%%%%%%%
%%%%%%%%%%%%%%%%%%%%%%%%%%%%%


%%%% Formatting Commands

\newcommand{\cimdia}[1] {\miniframesoff \begin{frame}\begin{center}\huge \begin{tabular}{c}#1\end{tabular}\end{center}\end{frame}\miniframeson}
\newcommand{\szakasz}[2][]{\section{#1}\subsection{}\cimdia{#2}}
\newcommand{\bluebullet}{\textcolor{DeepSkyBlue3}{\quad $\bullet$} \,\,}

\newenvironment{frame*}[1][]{\miniframesoff \begin{frame} #1}{\end{frame}\miniframeson}

  % for admissible intersections
  \newcommand{\bigsqcap}{\rotatebox[origin=c]{180}{$\bigsqcup$}}

\newcommand{\pecset}[2]{\begin{tikzpicture}[remember picture,overlay]
\node [ draw=red, rectangle, rounded corners=5mm, inner sep=1mm, ultra thick, fill=white, fill opacity=.8, rotate=30, scale=#1, text opacity=0.7] at (current page.center) {#2};\end{tikzpicture}}

\newcommand{\felirat}[7][]{\begin{tikzpicture}[remember picture,overlay]
\node [draw=DeepSkyBlue3, rectangle, rounded corners=#3 mm, inner sep=#2mm, ultra thick, fill=white, fill opacity=.8, scale=#4, text opacity=1,#1]
at ([xshift=#5 cm, yshift=#6 cm]current page.center) {#7};
\end{tikzpicture}}

\newcommand{\nobfelirat}[7][]{\begin{tikzpicture}[remember picture,overlay]
\node [rectangle, rounded corners=#3 mm, inner sep=#2mm, ultra thick, fill=white, fill opacity=.8, scale=#4, text opacity=1,#1]
at ([xshift=#5 cm, yshift=#6 cm]current page.center) {#7};
\end{tikzpicture}}

\newcommand{\hazi}[8][]{\begin{tikzpicture}[remember picture,overlay]
\node [ draw=Coral1,
        rectangle,
        rounded corners=#3 mm,
        inner sep=#2mm,
        ultra thick,
        fill=white,
        fill opacity=.8,
        rotate=0,
        scale=#4,
        text opacity=1, #1]
        at ([xshift=#5 cm, yshift=#6 cm]current page.center)
        {\begin{minipage}{#7}#8\end{minipage}};
\end{tikzpicture}}

\newcommand{\underconstruction}[1]{\begin{tikzpicture}[remember picture,overlay]
\node [rectangle, rounded corners=5mm, inner sep=1mm, rotate=30, scale=#1, text opacity=0.4]at (current page.center){\textsc{\textcolor{orange}{\begin{tabular}{c}under \\construction\end{tabular}}}};
\end{tikzpicture}}
\newcommand{\dzsa}[1]{\textsc{\underline{#1}}:}
\newcommand{\axiom}[1]{\bemph{(\mathrm{#1})}}



% Emphasizing:
\definecolor{barna}{rgb}{0.5,0.2,0.1}
\newcommand{\bemph}[1] {{\color{DeepSkyBlue3}{#1}}}
\newcommand{\kemph}[1] {{\color{blue}{#1}}}
\newcommand{\cemph}[1]{\textcolor{red}{#1}}
\newcommand{\zemph}[1] {{\color{Green2}{#1}}}
\newcommand{\yemph}[1] {{\color{Orange1}{#1}}}
%\renewcommand{\emph}[1]{\textbf{#1}}

\renewcommand{\Diamond}{\scalebox{.9}{\raisebox{-.4ex}{\rotatebox{45}{$\Box$}}}}

 \newcommand{\vonal} [1][.2]{\hspace{#1cm} | \hspace{#1cm}}

 \newcommand{\lrule}[3][c]{\begin{array}{#1} #2  \\  \hline #3 \end{array}}
 \newcommand{\dlrule}[3][c]{\begin{array}{#1} #2  \\  \hline\hline #3 \end{array}}
 \newcommand{\dual}{\delta}

 \newcommand{\mono}{\rightarrowtail}
 \newcommand{\epi}{\twoheadrightarrow}
 \newcommand{\iso}{\rightarrowtail \!\!\!\!\! \rightarrow}

 \newcommand{\defegy}[1][.1]{\hspace{#1cm}\overset{\textup{\tiny def}}{=}\hspace{#1cm}}
 \newcommand{\defpont}[1][.1]{\hspace{#1cm}\overset{\textup{\tiny def}}{:}\hspace{#1cm}}
 \newcommand{\defekv}[1][.1]{\hspace{#1cm}\overset{\textup{\tiny def}}{ \Leftrightarrow }\hspace{#1cm}}
 \newcommand{\kisiff}{\Leftrightarrow}
 \newcommand{\kisimplies}{\Rightarrow}
 \newcommand{\PB}{\mathbf H}
 \newcommand{\PBDot}{\underline{\mathbf H}}
% \newcommand{\implies}{\Longrightarrow}
 \newcommand{\lthen}{\rightarrow}
 \newcommand{\liff}{\leftrightarrow}
 \newcommand{\forallin}[2]{(\forall #1 \in #2)}
 \newcommand{\existsin}[2]{(\exists #1 \in #2)}
 \newcommand{\nexistsin}[2]{(\nexists #1 \in #2)}
 \newcommand{\forallp}[1]{(\forall #1)}
 \newcommand{\existsp}[1]{(\exists #1)}

\newcommand{\magyi}[1]{\textup{\bemph{\tiny #1}}}
\newcommand{\magyarazat}[2]{\overset{\substack{\textup{#2}\\ \downarrow}}{#1}}

\newcommand{\theory}[2][]{\mathrm{th}_{\mathfrak{#1}}(#2)}

\newcommand{\seenby}{\reflectbox {$R$}}
\newcommand{\derives}[1][]{\vdash_{\mathrm{#1}}}
\newcommand{\legkisebb}[2]{{[}\mathrm{min}\, #1{]} #2}
\newcommand{\legnagyobb}[2]{{[}\mathrm{max}\, #1{]} #2}

%%%%%%%%%%%%%%%%%%%%%%%%%%%%%%%%%%%%%%%%%%%%%%%%%%%%%%
\newenvironment{tomb}[2][.1]{\arraycolsep=#1cm\begin{array}{#2}}{\end{array}}

\beamertemplatenavigationsymbolsempty
\author{Attila Moln\'ar}
\date{2015. February 22.}
\title{Provability Logic}
\institute{ELTE}

\newcommand{\kicsinyit}[3][1]{\scalebox{#2}{\begin{minipage}{#1\textwidth}#3\end{minipage}}}


 	
%TÁRGYNYELV



%Metamatematikai jelek
	%Logika
		%Kondicionális
			\newcommand{\mlthen}[2]{ \begin{array}c  #1 \\ \hline #2 \end{array}}
		%Bikondicionális
			\newcommand{\mliff}[2]{ \begin{array}c  #1 \\ \hline  \hline #2 \end{array}}
		%Konjunkció
			\newcommand{\mland}[2]{\begin{array}l #1 \\ #2 \end{array}}
		%Negáció
			\newcommand{\mlnot}[1]{ \lnot #1}
	%Következményrelációk	
		%Szintaktikai következményreláció
			\newcommand{\lev}[2]{#1 \vdash #2}
		%Szemantikai következményreláció
			\newcommand{\kov}[2]{#1 \Rightarrow #2}
	%Szemantikai igazsággal kapcsolatos állítások
		\newcommand{\strue}[2]{\left| #2\right|_{#1} =i }
		\newcommand{\sfalse}[2]{\left| #2\right|_{#1} =h }
		%\newcommand{\strue}[2]{ \mathfrak{#1}\models #2}
		%\newcommand{\sfalse}[2]{\mathfrak{#1}\models\!\! \setminus #2}
		
% Definiáló egyenlőségek
			\newcommand{\struedef}[2]{\left| #2\right|_{#1} \defegy i }
			\newcommand{\sfalsedef}[2]{\left| #2\right|_{#1} \defegy h }
	%Helyettesítés
		\newcommand{\termsub}[3]{\left({#1}\right)\left[{#3}/{#2}\right]}
	%Univerzális séma:
		\newcommand{\us}{\mathcal}


%Halmazelméleti műveletek és relációk
	%Unió	
		\newcommand{\mlcup}[2]{ #1 \cup \, #2 }
	%Standard modell műveletei
		%Szukcesszor
			\newcommand{\mlsuc}[1]{ ({#1} \cup \{{#1} \} )}
		%Összeg
			\newcommand{\mlplus}[2]{ |{#1} \uplus {#2} | }
		%Szorzás
			\newcommand{\mltimes}[2]{ |{#1} \times {#2} |}
\newcommand{\csuszkazd}[1]{\only<2>{\hspace{-#1 cm}}}

\tikzset{
world/.style={inner sep=.5mm, fill=black, circle, draw=black, thick},
redworld/.style={inner sep=.5mm, fill=red, circle, draw=black, thick},
light/.style={>=stealth,->,decorate,decoration={pre length=0.2cm, post length=0.2cm, snake, amplitude=.4mm, segment length=2mm}},
time/.style={>=stealth,->,decoration={border,segment length=2mm, angle=90, amplitude=.5mm}, postaction={draw,decorate}},
tav/.style={>=stealth,<->, fill=red, draw=red},
causal/.style={>=stealth,->, opacity=.3, thick},
geometria/.style={opacity=.3},
}

\newcommand\prolog{\raisebox{-1ex}{\includegraphics[height=2em]{prologlogo.png}}}
\newcommand\ol[1]{\textcolor{blue}{\texttt{#1}}}
\newcommand\ml[1]{\textcolor{red}{\texttt{#1}}}
\newcommand\ha{:-}
\newcommand\valasz{\pause \hfill}

\begin{document}
\footnotesize
\begin{frame}
\centering
\textsc{\Large Prolog 2025 \\[1em] \textsc{elte}, logic department}

\bigskip

{ \small \begin{tabular}{c}
Attila Moln\'ar
\\Kőbányai Szent László Gimnázium
\\[3em] \includegraphics[width=2.5cm]{pseudobagoly.png}
    \end{tabular}
    }

 \begin{figure}
%\includegraphics[scale=.3]{elte_cimer.png}
%\includegraphics[scale=.3]{elte_cimer.png}
 \end{figure}

	\today
\end{frame}

\szakasz[Cut]{Cut}

\begin{frame}[t]
\frametitle{Cut}
\vspace{.5cm}

Prolog works with backtracing which can lead to inefficiency, i.e. wasting time and memory on possibilities that lead nowhere.
\vspace{.5cm}

We can control backtracking by the cut predicate: !/0
\vspace{.5cm}

Cut is a goal that always succeeds, so \prolog will always get through it.
\vspace{.5cm}


\ol{p(X):- b(X), c(X), d(X), e(X). }

\begin{center}
vs
\end{center}

\ol{p(X):- b(X), c(X), !, d(X), e(X). }



\end{frame}

\begin{frame}[t]
\frametitle{Cut}
\vspace{.5cm}

The cut only commits us to choices made since the parent goal was unified with the \cemph{left-hand side} of the clause containing the cut.
\vspace{.5cm}

\ol{q:- p$_1$,\dots , p$_m$, !, r$_1$, \dots , r$_n$. }
\vspace{.3cm}

when we reach the cut it commits us:
\begin{itemize}
\item to this particular clause of \ol{q}
\item to the choices made by \ol{p$_1$, \dots , p$_m$}
\item \cemph{NOT} to choices made by \ol{r$_1$, \dots , r$_n$}
\end{itemize}



\end{frame}

\begin{frame}[t]
\frametitle{Green cut}

Let's consider the following \ol{max/3} predicate:
\vspace{.3cm}

\ol{max(X,Y,Y):- X =< Y.}\\
\ol{max(X,Y,X):- X > Y.}
\vspace{1cm}
\pause

$\left[ \begin{minipage}{.27\textwidth}
\vspace{.1cm}
\ol{?- max(2,3,3)}\\
\pause
\ol{yes}\\

\ol{?- max(7,3,7)}\\
\pause
\ol{yes}
\vspace{.1cm}
\end{minipage}\right] $
\pause
$\left[ \begin{minipage}{.27\textwidth}
\vspace{.1cm}
\ol{?- max(2,3,2)}\\
\pause
\ol{no}\\

\ol{?- max(2,3,5)}\\
\pause
\ol{no}
\vspace{.1cm}
\end{minipage}\right] $
\pause
$\left[ \begin{minipage}{.27\textwidth}
\vspace{.1cm}
\ol{?- max(2,3,Y)}\\
\pause
\ol{Y = 3}\\

\ol{?- max(7,3,\_)}\\
\pause
\ol{true}
\vspace{.1cm}
\end{minipage}\right] $
\vspace{1cm}
\pause

After \ol{?- max(2,3,Y)} if asked for more solutions, polog will try to satisfy the second clause, which is completely useless -- since we know the two clauses are exclusive (hence, it returns with \ol{false}).

\end{frame}

\begin{frame}[t]
\frametitle{Green cut}

We can fix this by adding a cut to the rule:
\vspace{.5cm}

\ol{max(X,Y,Y):- X =< Y, !.}\\
\ol{max(X,Y,X):- X > Y.}
\vspace{.5cm}
\pause

Now
\begin{itemize}
\item If the \ol{X =< Y} \cemph{succeeds}, the cut commits us to this choice, and the \cemph{second clause} of \ol{max/3} is \cemph{not considered}.
\item If the \ol{X =< Y} fails, Prolog goes on to the second clause.
\end{itemize}

\end{frame}

\begin{frame}[t]
\frametitle{}


\centering
\begin{tikzpicture}
\node[anchor = west] at (-3, 1.5){\textsc{Without cut}};
\node[anchor = west] at (3, 1.5){\textsc{With cut}};
\node[anchor = west] at (-3, 0.7){\ol{max(X,Y,Y):- X =< Y.}};
\node[anchor = west] at (-3, 0.2){\ol{max(X,Y,X):- X > Y.}};
\pause
\node[overlay, anchor = north west] at (-3,-.3){\includegraphics[scale=.8]{1.png}};
\pause
\node[overlay, anchor = north west] at (-3,-3.3){\includegraphics[scale=.8]{2.png}};
\pause
\node[anchor = west] at (3, 0.7){\ol{max(X,Y,Y):- X =< Y, !.}};
\node[anchor = west] at (3, 0.2){\ol{max(X,Y,X):- X > Y.}};
\pause
\node[overlay, anchor = north west] at (3,-.7){\includegraphics[scale=.8]{3.png}};
\pause
\node[overlay, anchor = north west] at (3,-3.3){\includegraphics[scale=.8]{4.png}};

\end{tikzpicture}


\end{frame}

\begin{frame}[t]
\frametitle{Green cut, red cut}

\textsc{Green cut:} cuts that do not change the meaning of a predicate. \\
E.g. \ol{max/3} -- the new code gives exactly the \bemph{same answers} as the old version, but it is \bemph{more efficient}.\\
\vspace{.5cm}
\pause

To see how red cuts work, let's modify our \ol{max/3}:\\
\vspace{.25cm}

\ol{max(X,Y,Y):- X =< Y, !.}\\
\ol{max(X,Y,X):- X > Y.}
\vspace{.7cm}

The second clause is basically redundant. Let's remove it!
\vspace{.25cm}

\ol{max(X,Y,Y):- X =< Y, !.}\\
\ol{max(X,Y,X).}

\vspace{.5cm}

Now what?

\end{frame}

\begin{frame}[t]
\frametitle{Red cut}

\vspace{.5cm}


\ol{max(X,Y,Y):- X =< Y, !.}\\
\ol{max(X,Y,X).}
\vspace{.5cm}
 \pause

$\left[ \begin{minipage}{.27\textwidth}
\vspace{.1cm}
\ol{?- max(2,3,X)}\\
\pause
\ol{X = 3}
\vspace{.1cm}
\end{minipage}\right] $
\pause
$\left[ \begin{minipage}{.27\textwidth}
\vspace{.1cm}
\ol{?- max(4,3,X)}\\
\pause\ol{X = 4}
\vspace{.1cm}
\end{minipage}\right] $
\pause
$\left[ \begin{minipage}{.27\textwidth}
\vspace{.1cm}
\ol{?- max(2,3,2)}\\
\pause
\texttt{\textcolor{red}{true}}

\vspace{.1cm}
\end{minipage}\right] $
\vspace{.5cm}
\pause

Now \prolog will be able to answer alternative questions/queries, but not yes-no ones.
\vspace{.5cm}

Because of the form of the query, it will go directly to the second clause, and only check \ol{max(X,Y,X).}, which is obviously true -- syntactically.





\end{frame}


\begin{frame}[t]
\frametitle{Red cut}
\vspace{.5cm}

$\left[ \begin{minipage}{.45\textwidth}
\vspace{.1cm}
\ol{max(X,Y,Y):- X =< Y, !.}\\
\ol{max(X,Y,X).}
\vspace{.1cm}
\end{minipage}\right] $
$\left[ \begin{minipage}{.45\textwidth}
\vspace{.1cm}
\ol{?- max(2,3,2)}\\
\texttt{\textcolor{red}{true}}
\vspace{.1cm}
\end{minipage}\right] $

\vspace{.5cm}

\dots however, we can patch it with a unification after the cut:
\vspace{.25cm}

\ol{max(X,Y,Z):- X =< Y, !, Y=Z.}\\
\ol{max(X,Y,X).}
\vspace{.5cm}
\pause

Mind the \ol{Z} in the head of the rule! This \ol{(X,Y,Z)} tempts \prolog to try to substitute in the first clause, since \ol{(X,Y,Z)} under appropriate circumstances can unify with \ol{(X,Y,X)}!
\vspace{.25cm}
\pause

After \ol{X =< Y} (2 =< 3) succeeds, we ask if 3=2, which will be \cemph{false}.
\vspace{.25cm}
\pause

Because of the \ol{!} \prolog cannot go back and try another setup for substitution (\ol{max(X,Y,X)}), it will be stuck with what was given before the \ol{!}.
\vspace{.25cm}
\pause

However, if we ask \ol{max(4,3,X)}, it will be able to go to the second clause after failing with the first, since if fails already before the \ol{!}, at \ol{X =< Y}.

\end{frame}


\begin{frame}[t]
\frametitle{Red cut}

\vspace{.5cm}
\textsc{Red cut:} a cut that changes the meaning of the predicate. If we remove it, we do not get an equivalent program.
\vspace{1cm}
\pause
Using red cut the resulting program
\begin{itemize}
\item will not be fully declarative (we interfere with the control flow)
\item can be harder to read
\item might be prone to subtle programming mistakes
\end{itemize}

\end{frame}


\begin{frame}[t]
\frametitle{Negation as Failure}
\vspace{.2cm}

Combining \ol{!, fail.} we can express \cemph{negation}:\\
\vspace{.3cm}
\pause

\ol{burger(X):- bigMac(X).}\\
\ol{burger(X):- bigKahunaBurger(X).}\\
\ol{burger(X):- whopper(X).}
\vspace{.25cm}
\pause

\ol{bigMac(a).}\\
\ol{bigKahunaBurger(b).}\\
\ol{bigMac(c).}\\
\ol{whopper(d).}
\vspace{.25cm}
\pause

\ol{enjoys(vincent,X):- bigKahunaBurger(X),!, fail.}\\
\ol{enjoys(vincent,X):- burger(X).}
\vspace{.4cm}
\pause

\ol{?- enjoys(vincent,a).}\\
\pause
\ol{true}
\vspace{.25cm}
\pause

\ol{a} gets substituted in \ol{bugKahunaBurger(X)} where it will fail and go for the second clause. Among \ol{burger}s \ol{bigMac} is the first, so it succeeds, the query will be true.

\end{frame}

\begin{frame}[t]
\frametitle{Negation as Failure}
\vspace{.2cm}

Combining \ol{!, fail.} we can express \cemph{negation}:\\
\vspace{.3cm}

\ol{burger(X):- bigMac(X).}\\
\ol{burger(X):- bigKahunaBurger(X).}\\
\ol{burger(X):- whopper(X).}
\vspace{.25cm}

\ol{bigMac(a).}\\
\ol{bigKahunaBurger(b).}\\
\ol{bigMac(c).}\\
\ol{whopper(d).}
\vspace{.25cm}

\ol{enjoys(vincent,X):- bigKahunaBurger(X),!, fail.}\\
\ol{enjoys(vincent,X):- burger(X).}
\vspace{.3cm}

\ol{?- enjoys(vincent,b).}\\
\pause
\ol{false}
\vspace{.25cm}
\pause

Prolog tries to substitute \ol{b} in \ol{bigKahunaBurger(X)} in the first clause. It will succeed, since \ol{bigKahunaBurger(b).}, so it goes on, through \ol{!} all the way to \ol{fail}. Because of the \ol{!} \prolog cannot go back and look for other possible substitutions for \ol{b}\dots

\end{frame}

\begin{frame}[t]
\frametitle{Negation as Failure}
\vspace{.5cm}

The cut-fail combination offers us some form of negation.
\vspace{.3cm}

It is called \cemph{negation as failure}, and defined as follows:
\vspace{.35cm}

\ol{neg(Goal):- Goal, !, fail.}\\
\ol{neg(Goal).}
\vspace{.2cm}

\emph{If Goal succeeds, make the clause fail, otherwise let it go on.}
\vspace{.5cm}

Since negation as failure is frequently used in \prolog, there is a built-in predicate for it: \ol{$\setminus$+}


\end{frame}


\begin{frame}[t]
\frametitle{Negation as Failure}
\vspace{.5cm}

3 equivalent expressions of the fact that \ol{vincent enjoys} all sorts of \ol{burger}s except for \ol{bigKahunaBurger}:
\vspace{.4cm}

\ol{enjoys(vincent,X):- bigKahunaBurger(X), !, fail.}\\
\ol{enjoys(vincent,X):- burger(X).}
\vspace{.3cm}

\ol{enjoys(vincent,X):- burger(X), neg(bigKahunaBurger(X)).}\\
\vspace{.3cm}

\ol{enjoys(vincent,X):- burger(X), $\setminus$+bigKahunaBurger(X).}\\
\vspace{1cm}


Negation as failure is not a logical negation.
\vspace{.2cm}

After the \ol{!} we cannot go back, the program will fail.
\vspace{.2cm}

This is why we need to have \ol{burger(X)} before \ol{$\setminus$+}. We need to allow \prolog to freely select, substitute from our set of \ol{burger}s first.

\end{frame}


\szakasz[A trick with fail]{A trick with \ol{fail}}

\begin{frame}[t]
\frametitle{A trick with \ol{fail}}

Let's say we want \prolog not only to give us a suggestion about what \ol{burger}s \ol{vincent} would like, but we want to be able to have a full list of these \ol{burger}s!
\vspace{.25cm}
\pause

\ol{burger(X):- bigMac(X).}\\
\ol{burger(X):- bigKahunaBurger(X).}\\
\ol{burger(X):- whopper(X).}
\vspace{.2cm}

\ol{bigMac(a).}\\
\ol{bigKahunaBurger(b).}\\
\ol{bigMac(c).}\\
\ol{whopper(d).}
\vspace{.2cm}

\ol{enjoys(vincent,X):- burger(X), $\setminus$+bigKahunaBurger(X).}\\
\vspace{.2cm}
\pause

\ol{whattobuyto(Y):- enjoys(Y,X), write(X), nl, fail; true.}
\vspace{.25cm}
\pause

Until there are \ol{burger}s that have never been tested, \ol{fail} will make \prolog going back to \ol{burger}s -- since \prolog backtracks locally. At the point where all the \ol{burger}s inthe KB have been checked, \prolog will go back to the main rule (\ol{whattobuyto}) to see if there are any other ways to save the day, e.g. a second clause\dots

\end{frame}

\end{document}




\end{document} 