
\szakasz[Intro]{Introduction}

\begin{frame}[t]
\frametitle{Brief history}
\begin{itemize}
\item[1972] Marseille (PROgramming in LOGic)
\item[1978] PROLOG interpreter
\item[1987] SWI (free; online \& downloadable -- has better debugger). \[\prolog \defegy \textup{\textsc{swi prolog} interpreter}\]
\item[1987] Jaffar and Lassez: Constraint logic programming. Powerful and beautiful clp packages are included since.
\item[1990] SICStus Prolog (proprietary)
\end{itemize}

Used at\dots
\begin{itemize}
\item Watson -- Q\&A machine of IBM e.g. cancer
\item NASA -- international space station (Clarissa)
\item Ericsson -- Network Resource Manager (operator assistant)
\item Logistics
\item Data mining -- Rule Discovery System (RDS)
\end{itemize}
\end{frame}


\begin{frame}[t]
\frametitle{Declarative, logic programming}
\vspace{.5cm}

\textbf{Declarative programming:} define \textbf{what} to solve, not \textbf{how} to solve it.

\bigskip

We describe the problem (instead of the way to the solution -- algorithms), the program solves it using it's own strategies (e.g. backtracking)

\bigskip
\bigskip
\textbf{Logic programming:} syntax follows the syntax of logic (Horn logic).

\hspace{.8cm} Program: logical formula(s)

\hspace{.8cm} Running the program: evaluation of the formula(s)

\hspace{.8cm} No classical variables (temporal substitution vs assignment)

\hspace{.8cm} Recursion instead of loops

\begin{tikzpicture}
\node[overlay, rotate = 30] at (11,-2){\includegraphics[scale=.5]{prologlogo.png}};
\end{tikzpicture}
\end{frame}
