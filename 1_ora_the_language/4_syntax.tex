
\szakasz[Syntax]{Syntax}



\begin{frame}[t]
\frametitle{Language elements}
\framesubtitle{Constants}
	\begin{itemize}
		\item Atoms
			\begin{itemize}
			\item \footnotesize String beginning with lower case letter
            \begin{center} \ol{dEPARTMENT},\quad \ol{logic},\quad \ol{l\_o\_g\_i\_c},\quad \ol{year1988}\end{center}
			\item \footnotesize String in single quotes
\begin{center}
  \ol{'Mekis'}, \quad \ol{'\_String'},\quad  \ol{'WHAT3V3R'}
\end{center}
%			\item String of special characters \ol{;}, \ol{:-}
			\end{itemize}
		\item Numbers: usual\dots
\begin{center} \ol{-1}, \ol 0, \ol 1, \ol 2,\dots, integers, floats\dots\end{center}
	\end{itemize}
	
	\begin{tikzpicture}
\node[overlay, rotate = -30] at (-1,-2){\includegraphics[scale=.8]{prologlogo.png}};
\end{tikzpicture}
\end{frame}



\begin{frame}[t]
\frametitle{Language elements}
\framesubtitle{Variables}
		\begin{itemize}
		\item String starting with upper case letter
            \begin{center}\ol{Mekis}, \quad \ol{YOLO},\quad \ol{L\_AbamBA} \end{center}
		\item String starting with
            \begin{center}\ol{\_},\quad \ol{\_logic},\quad \ol{\_LoL} \end{center}
		\end{itemize}
The variable \ol{\_} is very special, \prolog will treat it differently!

(The string cannot contain any \ol{(}, \ol{)} or \ol{~}.)
\end{frame}

\begin{frame}[t]
\frametitle{Exercise}

Which of the following sequences of characters are \textbf{atom}s, which are \textbf{variable}s, and which are \textbf{neither}?
\vspace{1cm}

\centering
\begin{tabular}{ll}
\ol{thisiaVariable} & \pause atom \\
\ol{ELTE} & \pause variable \\
\ol{this\_is\_a\_Variable} & \pause atom \\
\ol{’\_John} & \pause neither \\
\ol{'John loves Mary'} & \pause atom \\
\ol{prolog2017}  & \pause atom \\
\ol{Nimbus2000} & \pause variable \\
\ol{John loves Mary} & \pause neither \\
\ol{\_John} & \pause variable \\
\ol{'John'} & \pause atom \\
\end{tabular}


\end{frame}

\begin{frame}[t]
\frametitle{Language elements}
\framesubtitle{Terms}
\begin{itemize}
\item Constants and variables are \textbf{terms}.
\item If $\ml f$ is an atom and \ml{$\tau_1$}, \dots, \ml{$\tau_n$} are \textbf{terms}, then
\[\ml f \ol{(}\ml {$\tau_1$} \ol{,} \ml{$\tau_2$} \ol , \ml{\dots} \ol ,\ml{$\tau_n$}\ol{)}  \]
is a (complex) \textbf{term}.

\[\ol{grandmotherof(X,motherof(fatherof(Y)))}\]

The number of contained terms are called the arities of the function/functor/predicate $\ml f$. In \textsc{prolog}, if $\ml f$ is a ternary function we refer to that in the following way: $\ml f \ol{/3}$. This will be very important, because it is not uncommon in \textsc{prolog} to use different functions such as $\ml f /2$ , $\ml f/3$ and $\ml f /5$ in a parallel way. We will never do that though.
\end{itemize}

	\begin{tikzpicture}
\node[overlay, rotate = 180] at (11,6.9){\includegraphics[scale=.4]{prologlogo.png}};
\end{tikzpicture}

\end{frame}


\begin{frame}[t]
\frametitle{Exercise}

 Which of the following sequences of characters are \textbf{atom}s, which are \textbf{variable}s, which are \textbf{complex term}s, and which are \textbf{not term}s at all? Give the functor and arity of each complex term.
 \vspace{1cm}

\centering
\begin{tabular}{ll}
\ol{’loves(Vincent,Mia)’} & \pause atom \\
\ol{loves(Vincent  Mia)}  & \pause not term \\
\ol{or(super(M),bat(M))} & \pause complex term (or/2, super/1, bat/1) \\
\ol{Butch(boxer)} & \pause neither \\
\ol{boxer(Butch)} & \pause complex term (butch/1) \\
\ol{beats(Batman,'Superman')} & \pause complex term (butch/2) \\
\ol{\_or(Super(M),Bat(M))}  & \pause neither \\
\ol{or(super(man),bat(man))} & \pause complex term (or/2, super/1, bat/1)\\
\ol{(Batman  beats  Superman)}  & \pause not term \\
\ol{loves(Vincent,Mia}  & \pause not term \\
\end{tabular}
\end{frame}


\begin{frame}[t]
\frametitle{Language elements}
\framesubtitle{Connectives}

\normalsize
\centering
\vspace{1cm}

\begin{tabular}{cc}
and & , \\
& \\
or & ; \\
& \\
if & :- \\
& \\
not & $\setminus$+
\end{tabular}

\bigskip
The ``not'' $\setminus$+ is \textbf{NOT} the negation you think of! Do not use it until we learn it how to use. It modifies the searching algorithm of \prolog, and as such it is not a declarative command!

%\begin{tikzpicture}
%    \node[overlay] at (5,1){\includegraphics[scale=.7]{prologlogo.png}};
%\end{tikzpicture}

\end{frame}

\begin{frame}
  \frametitle{Facts, rules and knowledge base}
If $\tau$ is a term \emph{that does not contain any variable}, then $\ml{$\tau$}\ol{.}$ is a \textbf{fact}. (Ending dot!)

\[\ol{loves(adam, motherof('Cain')).}\]

\begin{itemize}
\item If \ml{$\sigma$} and \ml{$\tau$} are terms, then \ml{$\sigma$}\ol\ha\ml{$\tau$}\ol{.} is \textbf{a rule}. \hfill (if)
\item If \ml{$\varrho$}\ol{.} is a \textbf{rule} and \ml{$\tau$} is a term, then \ml{$\varrho$}\ol{,}\ml{$\tau$}\ol{.} is a \textbf{rule} as well. \hfill(and)
%\item If \ml{$\varrho$} is a \textbf{rule} and \ml{$\tau$} is a term, then \ml{$\varrho$}\ol{;}\ml{$\tau$} is a \textbf{rule} as well. \hfill(or)
\end{itemize}

\[\ol{jealous(X, Y):-loves(X, Z),loves(Y, Z).}\]
\[\begin{array}{ll}
    \ol{jealous(\_loverNo1, \_loverNo2) :-}
        & \ol{loves(\_loverNo1, LovedOne),}
    \\  & \ol{loves(\_loverNo2, LovedOne).}\end{array}\]

(spaces and breaks do not matter)

\textbf{A knowledge base is a collection of facts and rules.}

\end{frame}

\begin{frame}
  \frametitle{Knowledge bases}
A \textbf{knowledge base} is collection of facts and rules.

\vfill

\noindent \ol{loves(vincent, mia).}
\\\ol{loves(marcellus, mia).}
\\\ol{loves(pumpkin, honey\_bunny).}
\\\ol{loves(honey\_bunny, pumpkin).}
\\\ol{jealous(X, Y) :- loves(X, Z),loves(Y, Z).}
\end{frame}

\begin{frame}
  \frametitle{Queries / proof searches}

\begin{tabular}{l}
\ol{loves(vincent, mia).}
\\\ol{loves(marcellus, mia).}
\\\ol{loves(pumpkin, honey\_bunny).}
\\\ol{loves(honey\_bunny, pumpkin).}
\\\ol{jealous(X, Y) :- loves(X, Z),loves(Y, Z).}
\\\hline
 \ol{jealous(X, Y).}
\\ \ml{and \prolog~will answer that...}
\end{tabular}


\end{frame}

\begin{frame}[t]
\frametitle{Exercise}

Translate the following rules into English (natural language)!
\vspace{.3cm}

\begin{tabular}{ll}
\textup{\ol{man('Eomund').}} & \textup{\ol{man('Eomer').}}\\
\textup{\ol{woman('Theodwyn').}} & \textup{\ol{woman('Eowyn').}}\\
\textup{\ol{son('Eomer', 'Eomund').}} & \textup{\ol{son('Eomer', 'Theodwyn').}} \\
\textup{\ol{daughter('Eowyn', 'Eomund').}} & \textup{\ol{daughter('Eowyn', 'Theodwyn').}}
\end{tabular}
\pause
\vspace{.3cm}


\ol{father(X,Y):- man(X), son(Y,X).}\\
\ol{father(X,Y):- man(X), daughter(Y,X).}\\
\pause
\ol{mother(X,Y):- woman(X), son(Y,X).}\\
\ol{mother(X,Y):- woman(X), daughter(Y,X).}\\
\pause
\ol{parent(X):- father(X,\_); mother(X,\_).}\\
\ol{parent(X,Y):- father(X,Y); mother(X,Y).}\\
\pause
\ol{sister(Y,Z):-parent(X),daughter(Y,X),daughter(Z,X),Z$\setminus$=Y.}\\
\ol{sister(Y,Z):- parent(X), daughter(Y,X), son(Z,X).}\\
\ol{brother(Y,Z):- parent(X), son(Y,X), son(Z,X), Z$\setminus$=Y.}\\
\ol{brother(Y,Z):- parent(X), son(Y,X), daughter(Z,X).}\\
\pause
\ol{sibling(X,Y):- brother(X,Y); sister(X,Y).}\\
\pause
\ol{ancestor(X,Y):- parent(X,Y).}\\
\ol{ancestor(X,Y):- parent(X,Z), ancestor(Z,Y).}\\

\end{frame}
